\title{%
  Brewery Boys \\
  \large Predicting Beer Ratings through Singular Value Decomposition and Collaborative Filtering \\
}
\author{
        Matt McFarland and Theodore Owens
}
\date{\today}


\documentclass[12pt]{article}
\renewcommand{\thesubsection}{(\alph{subsection})}

\usepackage[margin=1.0in]{geometry}
\usepackage{graphicx}
\usepackage{amsmath}
\usepackage{amssymb}
\usepackage{csquotes}
\usepackage{enumerate}
\usepackage{listings}
\usepackage{color}
\usepackage{booktabs}
\usepackage[toc,page]{appendix}
\usepackage{multirow}

\DeclareMathOperator*{\argmin}{arg\,min}

\renewcommand{\abstractname}{\vspace{-\baselineskip}}
\definecolor{mygray}{rgb}{0.9,0.9,0.9}
\lstset{ %
  backgroundcolor=\color{mygray},
  numbers=left,
  breaklines=true
}

\begin{document}
\maketitle
\begin{abstract}
% TODO
Using a dataset of beer reviews from \textbf{Beer Advocate}, we attempt to predict a reviewer's scoring of an unencountered beer based on tastes expressed through their previous reviews. We use two collaborative filtering approaches to make predictions: \textit{Singular Value Decomposition} and \textit{Item-to-Item Collaborative Filtering}. \\

We find that \textit{Singular Value Decomposition} can generate predicts 2\% better than the average baseline predictions. \textit{Item-to-Item Collaborative Filtering} produces a 1\% improvement compared to the baseline, conditional on limiting the dataset to users that have made many reviews. \\
\end{abstract}

\section{Preface}
To keep our terminology consistent with existing literature in collaborative filtering, we will take \textit{users} to mean reviewers on Beer Advocate and \textit{items} to mean the beers under review. For a user $i$ and an item $j$, let $Y_{ij}$ and $\hat Y_{ij}$ give the actual and predicted rating of that user on that item, respectively.

\section{Problem}
We are presented with $Y$, an $n \times d$ matrix of $n$ users and their ratings of $d$ beers. This matrix contains no other information about the users or beers and is very sparse, as most users have only rated a small subset of items. Our problem can be phrased as: \textbf{given a user $i$ and an beer $j$ that user $i$ has not rated, compute $\hat Y_{ij}$}.

%\paragraph{Collaborative Filtering}
We use Item-to-Item Collaborative Filtering and Single Value Decomposition to predict how a user will score a beer. These approaches predict review scores based on other user-item interactions in the dataset known to be similar to the target user and item. The accuracy of these prediction algorithms depends on having a large library of historical ratings to use as similarity references.\textsuperscript{\cite{sarwar}}

%Given the lack of feature information (categorization of items, demographics of users, etc.), we cannot rely on supervised learning techniques. We thus turn to unsupervised methods and collaborative filtering, in particular. Collaborative Filtering techniques attempt to establish similarities in user preferences for certain items based on observed user-item interactions. Predicted ratings rise when a user favorably rates items similar to the predicted item.

%\paragraph{Training} Our prediction for unseen user-item interactions depends on identifying similiar items to the target item. To establish item similarities for Item-Based Collaborative Filtering, we constructed a similiarity correlation matrix between all pairings of items. In the Single Value Decomposition Analysis, the decomposition of the user-item matrix reduces the feature dimensions of the item space into most-similar basis vectors.

%This process first requires establishing which items are similar to each other. Intuitively, if a user rates two items in the same fashion (both poorly or both well), then they are similar. If the user rates them oppositely, they are dissimilar. In our SVD analysis, we establish a set of latent features describing items that are similar in some respect, as determined by user preferences. For Item-Based Collaborative Filtering, we generate a correlation matrix that describes the similarity between all pairings of items.

%\paragraph{Prediction} Intuitively, if a user favorably rates item $A$, which is quite similar to an unrated item $B$, we expect the user to favorably rate item $B$. In the prediction stage, we make use of the user's provided ratings to make inferences about an unkown rating. In SVD analysis, we can predict the score of an unreviewed item by finding the product of that user's latent feature vector multiplied by that item's latent feature vector.

\section{Data}
Our dataset contains $1,586,599$ reviews concerning $n = 33,388$ users and $d = 65,680$ beers. Each review contains a primary rating on a scale from $0$ to $5$ by intervals of $0.5$. It also provides ratings in several other metrics (palate, taste, appearance, aroma) and information about the beer itself (brewery, style, ABV). For more information about the raw dataset, see the Appendix. We limit our analysis to predicting the primary rating (``overall review'').

%Each review contains a rating on a scale from $0$ to $5$ by intervals of $\frac{1}{2}$. It also provides user rating along several other metrics (palate, taste, appearance, aroma) as well as some information about the item itself (brewery, style, alcohol by volume).

\paragraph{Pre-Processing}
The dataset is comprised of a long list of ratings, where each row represents one user's rating of one beer. To generate our user-by-beer matrix $Y$, we rearranged these reviews into an $n$ by $d$ matrix, where each row represents a user and each column represents a beer in the catalog. The intersection between a user and a beer contains that user's review score of that beer if available.

To avoid noise and reduce computational complexity, we only include users who have reviewed at least 5 beers and beers that have at least 50 reviews. This leaves us with approximately $5,000$ items and $13,000$ users, where the resulting matrix is 1.8\% filled (but still represents the bulk of the data set with more than 1.2 million reviews included).

%We additionally remove all users who have not rated at least 5 items and all items that have not received at least 50 ratings. The denser the network of users and items, the better we can establish similarities amongst preferences and items. Removing obscure items and reviewers whose tastes are relatively unknown reduces the noise without hindering prediction quality. This leaves us with approximately $5,000$ items and $13,000$ users, where the resulting matrix is 1.8\% filled.

%Additionally, we form a modified user-by-item matrix $\mathbf{\bar X}$, where:

%$$ \mathbf{\bar X}_{ij} = \mathbf{X}_{ij} - \mu_{user_i} - \mu_{item_j}$$

%We derive the $\mu$ terms in our baseline algorithm. Intuitively, $\mathbf{\bar X}$ represents the residual of a rating that is not explained by the item's average (its ``quality'') and the user's rating tendencies.

\paragraph{Splitting Testing and Training Data} To partition the dataset into testing and training segments, we randomly separated out a percentage (15\% - 20\%) of all observed ratings and held those points from the training process. Our prediction models trained on the remaining set of reviews. We used the held-out set of testing data points to evaluate the predictive power of the trained model.

\paragraph{Error Measurement}
We used the Mean Squared Error (MSE) measurement to calculate the fit of our predictive model. While measuring training error, $S$ consists of the training data set. To calculate testing error, we used $S$ as the set of held-out testing data.

% $$ \mathbf{MAE} = \frac{\sum\limits_{i,j \epsilon S} \mathbf{\|Y_{ij} - \hat{Y}_{iy}\|}}{\sum\limits_{i,j \epsilon S}\mathbf{\|S\|} $$
$$ \mathbf{MSE} = \frac{\sum\limits_{i,j \in S} {\mathbf{|Y_{ij} - \hat{Y}_{ij}|}}^2}{{\mathbf{|S|}}^2} $$

\paragraph{Tools}
We implemented the data pre-processing and transformation of review list into a user-item interaction matrix with the Pandas library in Python. Item-to-Item Collaborative Filtering analysis was also done in Python. The Single Value Decomposition analysis was conducted in Matlab.

\section{Methods}
We use three methods to make predictions:

\begin{enumerate}
  \item \textbf{Baseline}: uses four types of rating means to establish baseline predictions against which we can compare our more advanced approaches.
  \item \textbf{Singular Value Decomposition}: uses feature reduction to expose principle features in the items and user preferences.
  \item \textbf{Item-to-Item Collaborative Filtering}: uses a pairwise correlation matrix of items and predicts based on ratings of similar items
\end{enumerate}

We additionally considered and ultimately dismissed implementing User-to-User Collaborative Filtering. This approach attempts to find similarities between users to make predictions. However, User-to-User Collaborative Filtering has been known to underperform compared to Item-to-Item Collaborative Filtering.\textsuperscript{\cite{sarwar}} This method also does not scale well with a large number of users because the similarity matrix must be recomputed with the addition of each new user. Given that it's easier for new users to join the dataset than to produce a new beer, it is better to build a prediction engine with user growth in mind.\textsuperscript{\cite{linden}}

\section{Baseline Predictors}
To establish a threshold of success for our algorithms, we first calculate a series of average baselines. The most basic predictor consists of predicting the global average rating $\mu_{global}$. Predicting the user's average rating $\mu_{user}$ and beer's average score $\mu_{item}$ are two other basic predictors.

Additionally, we implemented a baseline predictor used by Simon Funk in the Netflix Prize.\textsuperscript{\cite{Funk}} First, we calculate the mean for each item ($\mu_{item}$) for each column $j$. \textit{After} subtracting $\mu_{item}$ (each beer's average) from $Y$, we calculate the mean bias ($\mu_{bias}$) for each user above or below the beer average by averaging across row $i$. We then construct ${\mu}_{baseline_{ij}} = \mu_{bias_i} + \mu_{item_j}$ and use $\mu_{baseline_{ij}}$ as the predictor as $\hat{Y}_{ij}$.

%We then subtract the user bias from each row, reducing $Y$ to residuals. We form the predicted rating $\hat{Y}_{baseline_{ij}}$ for user $i$ on item $j$ by calculating:
%$$\hat{Y}_{baseline_{ij}} = {\mu}_{baseline_{ij}} = \mu_{bias_i} + \mu_{item_j}$$

\begin{table}[ht!]
\centering
\caption{Results of Mean Predictions}
\vspace{2mm}
\begin{tabular}{lllll}
\hline
\textbf{Predictor}         & $\mu_{global}$ & $\mu_{user}$ & $\mu_{item}$ & $\mu_{baseline}$ \\
\textbf{MSE}              & $0.4900$       & $0.4193$     & $0.3550$     & $0.3458$         \\ \hline
\end{tabular}
\end{table}

Because ${\mu}_{baseline}$ performs the best out of these baseline predictors, we will compare the predictive success of our machine learning algorithms against the error of this baseline.

\section{Singular Value Decomposition}
%\paragraph{Theoretical Basis} Single Value Decomposition factors an $n$ by $d$ matrix $Y$ into approximation matrices $U * \Sigma * V^{T}$, where $U$ is $n$ by $k$ and $V$ is $d$ by $j$. In our case, $Y$ is a large, sparse matrix of user-item interactions, and we can simplify the features of $Y$ by finding the primary latent features in $U$ and $V$ (where $\Sigma$ is a diagonal matrix multiplied into $U$ and $V$) for the $k$ largest eigenvectors. These latent features expose the directions of greatest variation, allowing us to identify items and users that are most similar in our dataset and build predictions based off of those similarities.

%\paragraph{Theoretical Basis} Single Value Decomposition is a form of Principal Component Analysis, which factors a matrix $Y$ into component matrices $U * \Sigma * V^{T}$. Principal Component Analysis alters the basis of $Y$ so that, in the factorized form, the directions of greatest variance are exposed in decreasing order. In our dataset, $Y$ represents the user-item interaction matrix, and when we decompose that matrix to constituent parts $U$ and $V$, we will find the eigenvectors of $Y$ that define the directions of greatest variance. Single Value Decomposition limits the factorization to the $K$ most significant eigenvectors. Because $Y$ is very large, but very sparse, we can simplify the data by factoring to $U$ and $V$. Where $U$ is a $n$ by $k$ matrix where each row represents a latent feature vector for a user, and $V$ is a $d$ by $K$ matrix where each row represents the latent feature vector for an item. These latent feature vectors are meaningful because, given an incomplete $Y$ matrix (where not all user-item interactions are accounted for), we can iteratively train $U$ and $V$ to approximate the data we do have for $Y$. New user-item interactions (ratings) are predicted with $\hat{Y}_{ij} = U_{i}V_{j}^{T}$.

%\paragraph{Previous Work} Single Value Decomposition has been used as a prediction mechanism for user-item interactions very successfully before, most famously in the Netflix Prize.\textsuperscript{\cite{gower}} Complex approaches, such as adaptively altering parameters like learning rate or incorporating time-sensitivity to a user's history, have been attempted to maximize the accuracy of predictions.\textsuperscript{\cite{ma}}\textsuperscript{\cite{gower}} We will attempt to adapt and apply some basic, successful techniques to our dataset.

%\paragraph{Literature Review} Single Value Decomposition as a technique to simplify a large, sparse user-interaction matrix has been used very successfully before, most famously in the famous Netflix Prize\textsuperscript{\cite{gower}}. Studies on the application of SVD to decompose a matrix has revealed several variants on the basic algorithm to increase the accuracy of the predictions. Such methods include adaptively altering the learning rate, changing the iterative update method, and adding regularization\textsuperscript{\cite{ma}}. Other have also added different normalization constants to the SVD analysis to account for quality and user biases\textsuperscript{\cite{paterek}}. Simon Funk, a famous contestant in the Netflix algorithm further tuned the bias weights based on the number of reviews an item or user had \textbf{\cite{Funk}}. Gower explains that successful Netflix SVD algorithms also incorporated time-based information about the user to increase the prediction's accuracy\textsuperscript{\cite{gower}}. The SVD method has been explored extensively, and we will attempt to replicate some of the most successful methods on our dataset.

\subsection*{Algorithm}
\paragraph{Training} The goal of the SVD analysis is to factor $Y$ into $U$ ($n \times k$) and $V$ ($d \times k)$ matrices whose product ($U V^T$) well appoximates $Y$, and where $k$ is the number of $Y$'s largest eigenvalues incorporated into $U$ and $V$.

To maximize the potential predictive power of this approach, we fit $U$ and $V$ to the residuals of the training data (matrix $R$) after subtracting the $Y_{baseline}$ prediction from $Y$.

 $$ R_{ij} = Y_{ij} - Y_{baseline_{ij}} $$

 By centering the data on $Y_{baseline}$, we can regularize our predictive error function to avoid overfitting. (Fitting $U$ and $V$ to non-centered data produced less accurate results. See the results section.) To find a $U$ and $V$ matrix that approximates $R$, we minimized the difference between the observed training residuals and the predicted residuals.

$$ [\hat{U}, \hat{V}] \leftarrow \argmin\limits_{U,V}\sum\limits_{i,j \epsilon S} {\|R_{ij} - U_i V_j^T\|}^2 + w_U\|U\|^2 + w_V\|V\|^2$$

Because $Y$ includes missing data, we cannot solve for the closed form solution and must train $U$ and $V$ with a gradient descent minimization of the prediction difference function. $S$ is the training set of observed beer ratings. \\

\begin{center}
$ \frac{\partial E}{\partial U_i} = -2 * \sum\limits_{i,j \epsilon S} (R^{ij} - U_i V_j^T) V_j + w_U U_i $ \hspace{1cm} $ \frac{\partial E}{\partial V_j} = -2 * \sum\limits_{i,j \epsilon S} (R^{ij} - U_i V_j^T) U_i + w_V V_j$\\
\end{center}

% \begin{center}
%   In compact matrix form, this is $\frac{\partial E}{\partial U} = -2 * I (Y - U V^T) U + w_U U$
% \end{center}

%$$ \frac{\partial E}{\partial V_j} = -2 * \sum\limits_{i,j \epsilon S} (R^{ij} - U_i V_j^T) U_i + w_V V_j$$

% \begin{center}
%   In compact matrix form, this is $\frac{\partial E}{\partial V} = -2 * (I (Y - U V^T))^T V + w_V V$
% \end{center}

Where $U$ and $V$ were updated with schotastic gradient descent (and where $\lambda$ is the learning rate / step size). While training $U$ and $V$, this update process continued until the error started to increase.
\begin{center}
$ U_{t+1} = U_t - \lambda * \frac{\partial E}{\partial U}$ \hspace{2cm} $V_{t+1} = V_t - \lambda * \frac{\partial E}{\partial V}$
\end{center}
\paragraph{Prediction} To predict a residual for user $i$ and beer $j$, we take the product of $U_i$ and $V_j^T$. By adding the baseline prediction to the predicted residual, we get the overall predicted score.
$$ \hat{R}_{ij} = U_i V_j^T $$
$$ \hat{Y}_{ij} = Y_{baseline_{ij}} + \hat{R}_{ij} $$

\subsection*{Hyper-Parameter Tuning}
The SVD prediction model required four hyperparameters: The learning rate ($\lambda$), the regularization weights for $U$ ($w_U$) and $V$ ($w_V$), and $k$ which determines the number of $Y$ eigenvectors included in $U$ and $V$. In each training run, we selected the largest $\lambda$ possible without causing divergence (varying from $0.5$ to $.001$). While determining the best weights and $k$, we trained and tested on a smaller subset of the the full dataset ($7,000$ users and $800$ beers) for time's sake.

\paragraph{Selecting Regularization Weights} We tested a matrix of regularization weights to determine which combination of \textit{light} ($\sim 0.02$) and \textit{heavy} ($\sim 30.0$) weights produced the best testing results. The results are included in Table \ref{tab:weight_table}. We selected the regularization weights $w_U = 20.0$ and $w_V = 10.0$ as the best weight combination, because larger weights produced more accurate predictions by preventing over training. We weighted $U$ more than $V$ to allow the training process to develop slightly more complex profiles for the items instead of the users. 

% Please add the following required packages to your document preamble:
% \usepackage{multirow}
\begin{table}[ht!]
\centering
\caption{Testing MSE Error for Different $w_U$ and $w_V$}
\label{tab:weight_table}
\begin{tabular}{lcccl}
                                           & \multicolumn{1}{l}{}       & \multicolumn{2}{c}{$w_U$}                                &  \\ \cline{3-4}
                                           & \multicolumn{1}{c|}{}      & \multicolumn{1}{c|}{light} & \multicolumn{1}{c|}{\textbf{heavy}} &  \\ \cline{2-4}
\multicolumn{1}{l|}{\multirow{2}{*}{$w_V$}} & \multicolumn{1}{c|}{light} & 0.3485                     & 0.3261                     &  \\ \cline{2-2}
\multicolumn{1}{l|}{}                      & \multicolumn{1}{c|}{\textbf{heavy}} & 0.3245                     & \textbf{0.3233}                      &  \\ \cline{2-2}
\end{tabular}
\end{table}

\paragraph{Selecting k} Having decided on a non-divergent $\lambda$ and weights $w_U$ and $w_V$, we then executed a form of cross validation to find the best $k$. The results are included in Table \ref{tab:k_selection}. The Avg MSE is the averaged MSE of the three validation tests for the given $k$. In the validation test, 20\% of the training data was randomly partitioned into a validation set. SVD trained on the remaining portion of the training set, and validation error was calculated with the found $U$ and $V$ on the validation data. Cross validation results showed that $k = 3$ generated the lowest average MSE. 

\begin{table}[ht!]
\centering
\caption{Cross Validation of k Results}
\label{tab:k_selection}
\vspace{2mm}
\begin{tabular}{llllllll}
\hline
\textbf{k}             &  1 				& \textbf{3}          	& 5             & 7             & 10        & 15        & 20 \\ \hline
\textbf{Avg MSE}       & $0.3140$		& {$\mathbf{0.3084}$}   	& $0.3103$      & $0.3110$      & $0.3116$  & $0.3124$  & $0.3148$\\ \hline
\end{tabular}
\end{table}

\subsection*{SVD Results}
Initially, we tried to fit $U$ and $V$ to the raw scores instead of to the residuals (with the found best $k = 3$). Predicting the raw score $\hat Y_{ij}$ as the product of $U_i$ and $V_j$ resulted in a testing error of $0.4584$ which was unacceptably higher than the best baseline.
%(See Figure \ref{fig:non_centered_unregularized} in the Appendix).

To improve performance, we altered our training algorithm to fit $U$ and $V$ to the residual matrix $R$, and summed the predicted residual and predicted baseline to get the review prediction. Using the found best $k = 3$, regularization weights $w_U = 20.0$ and $w_V = 10.0$, and learning rate $\mu = .5$, we trained the SVD on the full dataset of review residuals (reserving 15\% of the data as testing data). We were able to realize an MSE of $0.3375$ which represented an improvement of $2.4\%$ over the bias baseline predictor (MSE of $0.3458$). See Figure \ref{fig:svd_results}.

\begin{figure}[!ht]
\begin{center}
\caption{SVD Best Prediction Results}
\label{fig:svd_results}
    \includegraphics[width=.7\textwidth]{"./figures/final".jpg}
\end{center}
\end{figure}

\section{Item-to-Item Collaborative Filtering}

\subsection*{Algorithm}
\paragraph{Training} For this method, we also start from the residual set of ratings of $R$, where the user and item biases have been removed. We generate a $d \times d$ correlation matrix $C$ from $R$. The entry ${C}_{ij}$ describes the similarity of residuals between items $i$ and $j$. We use the Pearson Correlation to determine these similarity scores, which fall in the range $[-1, 1]$.

% If users rate beers $i$ and $j$ consistently well or poorly, the similarity increases. If users rate one postively and the other negatively, the similarity decreases.

Our prediction step requires knowing which items should be considered ``similar'' enough to each other to use as a basis for a prediction. We must discretize ${C}$ such that correlation scores above a certain threshold $s^*$ are considered ``similar''.\textsuperscript{\cite{sarwar}} By applying this threshold to all entries in ${C}$, we generate a $d \times d$ matrix ${S}$, where:

$$
\mathbf{S}_{ij} =
\begin{cases}
    \hfill 1    \hfill & \text{ if items $i$ and $j$ are similar ($C_{ij} > s^*$)} \\
    \hfill 0    \hfill & \text{ otherwise} \\
\end{cases}
$$

\paragraph{Prediction} Armed with the similarity matrix ${S}$, we can make predictions. For a given user $i$, we wish to predict his rating on an item $j$ that he has not yet rated, given his past ratings. Letting $S$ be a set of items similar to the predicted item $j$ that the user has also rated, We predict $\hat Y_{ij}$, where:

$$ \hat Y_{ij} = \mu_{baseline_{ij}} + \frac{\sum\limits_{s \in S} R_{is}}{|S|}$$

We predict our baseline plus an Item-to-Item Collaborative Filtering term.\textsuperscript{\cite{gower}} This term sums the users ratings for items similar to $j$ and divides by the number of similar items that the user has rated (takes an average).

\subsection*{Hyper-Parameter Tuning}
Item-to-Item Collaborative Filtering requires setting $s^*$ to a threshold to determine whether two items are similar or not. In Figure \ref{fig:similarity_tuning}, we calculate the prediction error for various similarity thresholds. We observe that this algorithm has the lowest prediction error where $s^* = 0$. As the $s^*$ increases, we have fewer similiar items from which to generate the collaborative filtering term, leading to greater variation in the ``average of similar items''. Even though non-correlated items are viewed as ``similar'' when $s^* = 0$, our results improve on the best baseline by removing all \textit{dissimilar} items from the comparison.

\begin{figure}[!ht]
\centering
    \includegraphics[width=.7\textwidth]{"./images/similarity_tuning".png}
    \caption{Prediction Error for Similarity Thresholds}
    \label{fig:similarity_tuning}
\end{figure}

\subsection*{Item-to-Item Collaborative Filtering Results}

Given the dependency of Item-to-Item Collaborative Filtering on a user's ratings of similar items, we run the algorithm against two datasets. The first dataset is consistent with our approach in SVD, and considers items with at least 50 ratings and users that have given at least 5 ratings. Under these conditions, we cannot with any regularity beat the baseline.

We restrict the second dataset to users that have made at least 500 ratings (with the same item requirement of 50 ratings). With this restriction in place, we find a modest 1\% improvement compared to our baseline.

\begin{table}[ht!]
\centering
\caption{Item-to-Item Collaborative Filtering Results}
\label{my-label}
\begin{tabular}{lllll}
\hline
                        & Baseline (Test) & CF (Test) & Baseline (Train) & CF (Train) \\ \hline
Unrestricted Data     & 0.3629             & 0.3440                  & 0.3304              & 0.2359                   \\
User Restricted Data  & 0.3045             & 0.3011                  & 0.3042              & 0.1918                   \\ \hline
\end{tabular}
\end{table}

When the feature matrix is denser (as in the user restricted dataset), our collaborative filtering approach strengthens relative to the baseline. There is a greater chance that any given item is both similar to the current predicted item \textit{and} the user has rated that similar item.

We also do note from testing our method against the training set that the collaborative filtering approach yields results that are significantly better than the baseline, suggesting a degree of overfitting. Regularization of the similarity matrix might alleviate the overfitting issue.

\section{Comparison of Methods}
We have demonstrated how both Single Value Decomposition and Item-to-Item Collaborative Filtering result in modest improvements compared to the best baseline prediction. Given how little variance exists in the distribution of reviews (see the Appendix section of data characterization), we consider any improvement on the best baseline to be a success. Comparing the two prediction methods against one another, we find SVD advantageous due to its feature reduction principle.

In SVD, we expose latent features that group the items along certain unkown criteria based on correlations in user ratings. By simplifying items and users into similar groups, we dimensionally reduce the enormity of the beer catalog and user population to a few stereotypes that illuminate common traits between beers and users.

Oppositely, Item-to-Item Collaborative Filtering relies only on direct comparisons between two items. Given that the number of pairings grows with the square of the number of items, we require a vast number of comparisons to make meaningful observations about the similarity of the two items.

Further, in the prediction stage, Item-to-Item Collaborative Filtering only takes advantage of knowledge drawn from ``similar'' items. However, knowing a user's opinion of \textit{dissimilar} items could be useful as well. If a user likes a dissimilar item, we may infer the user will dislike the predicted item.

So while both methods can contribute to more accurate predictions beyond the best baseline, SVD is superior in its ability to find defining common traits among users and beers whereas Item-to-Item Collaborative Filtering can only examine items known to be similar to each other.

In futher exploration, it would be interesting to see how SVD and Item-to-Item Collaborative Filtering could be used to augment one another's predictions in an ensemble prediction. Understanding where each method is uniquely strong and weak would show how to combine the method the produce more robust predictions overall.

\section{Contributions}
Matt implemented SVD algorithm and worked with Matlab. Ted focused on pre-processing features in Python and explored the Item-to-Item Collaborative Filtering approach. We each wrote the sections of the report that are relevant to our respective algorithms. Ted drafted opening and closing remarks, edited and finalized by Matt.

\newpage
\begin{thebibliography}{9}

\bibitem{Funk}
Funk, Simon, ``Netflix Update: Try This at Home'' The Evolution of Cybernetics. Web. 11 Dec. 2006.

\bibitem{gower}
Gower, Stephen. ``Netflix Prize and SVD.'' (n.d.): n. pag. 18 Apr. 2014. Web. 9 Mar. 2016.

\bibitem{linden}
Linden, Greg, Brent Smith, and Jeremy York. ``Amazon.com Recommendations: Item-to-item Collaborative Filtering.'' IEEE Internet Computing IEEE Internet Comput. 7.1 (2003): 76-80. Web.

\bibitem{ma}
Ma, Chih-Chao. ``A Guide to Singular Value Decomposition for Collaborative Filtering.'' (n.d.): n. pag. Depart of Computer Science, National Taiwan University. Web. 9 Mar. 2016.

\bibitem{paterek}
Paterek, Arkadiusz. ``Improving Regularized Singular Value Decomposition for Collaborative Filtering.'' Institute of Informatics, Warsaw University. Web. 12 Aug. 2007.

\bibitem{sarwar}
Sarwar, Badrul, George Karypis, Joseph Konstan, and John Reidl. ``Item-based Collaborative Filtering Recommendation Algorithms.'' Proceedings of the Tenth International Conference on World Wide Web - WWW '01 (2001): n. pag. Web.

\end{thebibliography}

\newpage
\begin{appendices}
\section*{Dataset Characterization}

    \begin{figure}[!ht]
    \begin{center}
    \caption{Sample Ratings}
        \includegraphics[width=.9\textwidth]{"./images/dataset_sample".png}
    \end{center}
    \end{figure}

    \begin{table}[ht!]
    \centering
    \caption{Whole Dataset Summary Statistics}
    \begin{tabular}{ll}
    \hline
    Number of Reviews         & 1,586,599 \\
    Number of Items           & 65,680    \\
    Number of Users           & 33,388    \\
    Rating Minimum            & 5.0       \\
    Rating Maximum            & 0.0       \\
    Rating Mean               & 3.82      \\
    Rating Variance           & 0.52      \\
    Rating Standard Deviation & 0.72      \\ \hline
    \end{tabular}
    \end{table}

    \begin{figure}[!ht]
    \begin{center}
    \caption{Rating Distribution}
        \includegraphics[width=.5\textwidth]{"./images/rating_hist".png}
    \end{center}
    \end{figure}

    \begin{figure}[!ht]
    \caption{Distributions of Number of Ratings by Item and by User}
    \centering
    \begin{minipage}{.5\textwidth}
      \centering
      \includegraphics[width=0.95\linewidth]{"./images/item_hist".png}
    \end{minipage}%
    \begin{minipage}{.5\textwidth}
      \centering
      \includegraphics[width=0.95\linewidth]{"./images/user_hist".png}
    \end{minipage}
    \end{figure}

% \clearpage
% \section*{SVD Results}
%     \begin{figure}[!ht]
%     \begin{center}
%     \caption{SVD Performance on Uncentered Data}
%         \includegraphics[width=.8\textwidth]{"./figures/non_centered_unregularized".jpg}
%         \label{fig:non_centered_unregularized}
%     \end{center}
%     \end{figure}


\end{appendices}
\end{document}
